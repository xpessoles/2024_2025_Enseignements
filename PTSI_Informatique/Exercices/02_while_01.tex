

\section*{Exercices - Boucles for}
\subsection{Niveau débutant}
\begin{enumerate}

    \item \textbf{Exercice 1} \\
    Utilisez une boucle \texttt{while} pour afficher les nombres de 1 à 10.

    \item \textbf{Exercice 2} \\
    Créez une boucle \texttt{while} qui affiche les nombres pairs de 2 à 20.

    \item \textbf{Exercice 3} \\
    Utilisez une boucle \texttt{while} pour afficher les multiples de 3, de 3 à 30 inclus.

    \item \textbf{Exercice 4} \\
    Écrivez une boucle \texttt{while} qui affiche tous les nombres de 10 à 1, dans l'ordre décroissant.

    \item \textbf{Exercice 5} \\
    Utilisez une boucle \texttt{while} pour calculer et afficher la somme des nombres de 1 à 50.

    \item \textbf{Exercice 6} \\
    Utilisez une boucle \texttt{while} pour calculer et afficher la somme des nombres pairs entre 1 et 100.

    \item \textbf{Exercice 7} \\
    Écrivez une boucle \texttt{while} pour calculer et afficher le produit des nombres de 1 à 10.

    \item \textbf{Exercice 8} \\
    Créez une boucle \texttt{while} qui calcule la puissance de 2, de \(2^1\) à \(2^{10}\), et affiche chaque résultat.

    \item \textbf{Exercice 9} \\
    Utilisez une boucle \texttt{while} pour afficher la suite de Fibonacci, jusqu'au 10\textsuperscript{ème} terme.

    \item \textbf{Exercice 10} \\
    Utilisez une boucle \texttt{while} pour afficher les nombres de 1 à 100. Si un nombre est multiple de 3, affichez \texttt{Fizz} à la place, et si un nombre est multiple de 5, affichez \texttt{Buzz}. Si un nombre est multiple de 3 et de 5, affichez \texttt{FizzBuzz}.

\end{enumerate}
