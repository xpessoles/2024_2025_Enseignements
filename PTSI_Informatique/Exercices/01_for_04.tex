

\subsection{Niveau 4}

\begin{enumerate}

    \item Écrivez un programme qui trie une liste d'entiers en utilisant l'algorithme de tri par insertion. Ne pas utiliser les fonctions de tri intégrées de Python, mais seulement des boucles \texttt{for} et des comparaisons.

    \item    Créez un programme qui prend une matrice (liste de listes) et calcule la somme des éléments de la diagonale principale (du coin supérieur gauche au coin inférieur droit). Utilisez une boucle \texttt{for} pour accéder aux éléments de la diagonale.

    \item Écrivez un programme qui génère tous les sous-ensembles possibles d'une liste donnée, sans utiliser de bibliothèques ou de fonctions prédéfinies pour le faire. Par exemple, pour \texttt{[1, 2, 3]}, le programme devrait générer \texttt{[[], [1], [2], [3], [1, 2], [1, 3], [2, 3], [1, 2, 3]]}.

    \item Créez un programme qui vérifie si deux chaînes de caractères sont des anagrammes (contiennent les mêmes lettres en quantités identiques, sans ordre particulier). Utilisez des boucles \texttt{for} pour compter et comparer les lettres de chaque chaîne.

    \item Écrivez un programme qui implémente l'algorithme de tri rapide (quick sort) en utilisant des boucles \texttt{for} et des fonctions récursives. Ne pas utiliser les fonctions de tri intégrées.

    \item Créez un programme qui prend une liste de chaînes de caractères et regroupe celles qui sont des anagrammes. Par exemple, pour \texttt{['rat', 'tar', 'art', 'bat', 'tab']}, le programme devrait retourner \texttt{[['rat', 'tar', 'art'], ['bat', 'tab']]}.

    \item Écrivez un programme qui utilise une boucle \texttt{for} pour générer les nombres de la suite de Fibonacci jusqu'à ce qu'un terme dépasse une valeur donnée par l'utilisateur. Affichez chaque terme généré.

    \item Créez un programme qui, donné un nombre entier positif, vérifie si ce nombre est un nombre parfait (c'est-à-dire égal à la somme de ses diviseurs propres). Par exemple, 6 est parfait car \texttt{1 + 2 + 3 = 6}.

    \item Écrivez un programme qui prend une matrice carrée (liste de listes) et vérifie si elle est symétrique (égale à sa transposée). Utilisez une boucle \texttt{for} pour comparer les éléments de la matrice.

    \item Créez un programme qui génère le triangle de Pascal jusqu'à une hauteur donnée. Utilisez une boucle \texttt{for} pour calculer chaque ligne du triangle. Par exemple, pour une hauteur de 5, le triangle doit ressembler à :
    \begin{verbatim}
        1
       1 1
      1 2 1
     1 3 3 1
    1 4 6 4 1
    \end{verbatim}

\end{enumerate}
