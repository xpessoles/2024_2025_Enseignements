\begin{enumerate}

    \item Écrivez un programme qui prend une liste de mots et affiche le mot le plus long de cette liste. Utilisez une boucle \texttt{for} pour parcourir la liste.

    \item Écrivez un programme qui affiche les nombres premiers de 1 à 100. Utilisez une boucle \texttt{for} imbriquée pour vérifier si chaque nombre est divisible uniquement par 1 et par lui-même.

    \item Créez un programme qui prend un mot et vérifie s’il est un palindrome (mot qui se lit de la même manière à l’envers). Utilisez une boucle \texttt{for} pour comparer les caractères.

    \item Écrivez un programme qui calcule le produit scalaire de deux listes d'entiers de même longueur. Par exemple, si \texttt{A = [1, 2, 3]} et \texttt{B = [4, 5, 6]}, le produit scalaire est \texttt{1*4 + 2*5 + 3*6 = 32}.

    \item Créez un programme qui génère une pyramide de nombres. Par exemple, si l'utilisateur entre \texttt{4}, le programme doit afficher : 
    \begin{verbatim}
    1
    22
    333
    4444
    \end{verbatim}

    \item Écrivez un programme qui utilise une boucle \texttt{for} pour trier une liste d’entiers en utilisant l'algorithme de tri par sélection. Ne pas utiliser de fonctions de tri intégrées.

    \item Créez un programme qui prend une liste de phrases et affiche les mots de chaque phrase qui contiennent exactement 5 lettres. Utilisez une boucle \texttt{for} pour parcourir chaque mot.

    \item Écrivez un programme qui utilise une boucle \texttt{for} pour compter le nombre de fois où chaque lettre apparaît dans une phrase donnée. Affichez le résultat sous forme de dictionnaire où chaque lettre est une clé.

    \item Créez un programme qui prend une liste de listes (par exemple \texttt{[[1, 2], [3, 4], [5, 6]]}) et calcule la somme de tous les éléments. Utilisez des boucles \texttt{for} imbriquées pour parcourir la liste principale et les sous-listes.

    \item Écrivez un programme qui génère les \texttt{n} premiers termes de la séquence de Fibonacci et les stocke dans une liste. Utilisez une boucle \texttt{for} pour générer les termes et afficher la liste finale.

\end{enumerate}


