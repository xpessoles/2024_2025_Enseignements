
\begin{enumerate}

    \item \textbf{Trouver la sous-liste ayant la plus grande somme} \\
    Écrivez un programme qui demande une liste de nombres et trouve la sous-liste continue ayant la somme maximale (sous-problème de la somme maximale). Utilisez une approche optimisée, comme l’algorithme de Kadane.

    \item \textbf{Réorganiser la liste en alternant des valeurs plus grandes et plus petites} \\
    Écrivez un programme qui réorganise une liste de manière à alterner des valeurs plus grandes et plus petites. Par exemple, transformez \texttt{[1, 2, 3, 4, 5, 6]} en \texttt{[2, 1, 4, 3, 6, 5]}.

    \item \textbf{Multiplication cumulée} \\
    Écrivez un programme qui prend une liste de nombres et crée une nouvelle liste où chaque élément est le produit cumulatif de tous les éléments précédents, sans utiliser de boucles imbriquées. Par exemple, pour la liste \texttt{[1, 2, 3, 4]}, la sortie doit être \texttt{[1, 1, 2, 6]}.

    \item \textbf{Trouver tous les triplets qui forment une somme nulle} \\
    Écrivez un programme qui demande une liste de nombres et trouve tous les triplets uniques de nombres dans la liste dont la somme est égale à zéro. Assurez-vous que les triplets sont uniques, même si les nombres sont répétés dans la liste.

    \item \textbf{Remplir la liste des valeurs manquantes pour en faire une liste consécutive} \\
    Écrivez un programme qui prend une liste de nombres non triés et la remplit avec les nombres manquants pour qu’elle devienne une séquence consécutive, de manière croissante, sans changer l'ordre des autres éléments.

    \item \textbf{Évaluation d'expressions en notation polonaise inverse (RPN)} \\
    Écrivez un programme qui prend en entrée une expression mathématique en notation polonaise inverse (Reverse Polish Notation) sous forme de liste, puis l'évalue et retourne le résultat. Par exemple, pour l'entrée \texttt{[2, 1, "+", 3, "*"]}, la sortie doit être $9$.

    \item \textbf{Liste des carrés sans duplications} \\
    Écrivez un programme qui prend une liste de nombres et renvoie une nouvelle liste contenant les carrés des nombres de la liste d'origine, sans doublons. Par exemple, pour l'entrée \texttt{[1, -1, 2, -2, 3]}, la sortie doit être \texttt{[1, 4, 9]}.

    \item \textbf{Détecter une sous-liste palindrome} \\
    Écrivez un programme qui prend une liste et détermine si elle contient une sous-liste continue qui est un palindrome. Par exemple, la liste \texttt{[3, 5, 6, 7, 6, 5, 3]} contient une sous-liste palindrome.

    \item \textbf{Équilibrer la somme des sous-listes} \\
    Écrivez un programme qui prend une liste et trouve un indice où la somme des éléments de gauche est égale à la somme des éléments de droite. Si cet indice n'existe pas, retournez \texttt{-1}. 

    \item \textbf{Intercaler deux listes} \\
    Écrivez un programme qui prend deux listes de longueurs différentes et crée une nouvelle liste qui intercale leurs éléments. Si une des listes est plus longue, placez les éléments restants de cette liste à la fin.

\end{enumerate}
