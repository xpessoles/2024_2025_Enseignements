
\begin{enumerate}

    \item Écrivez un programme qui affiche les nombres de 1 à 10 en utilisant une boucle \lstinline{for}.

    \item Écrivez un programme qui affiche tous les nombres pairs de 1 à 20.

    \item Créez un programme qui affiche les carrés des nombres de 1 à 10.

    \item Écrivez un programme qui affiche les éléments d'une liste de fruits \lstinline{['pomme', 'banane', 'orange', 'raisin']}.

    \item Écrivez un programme qui calcule et affiche la somme des nombres de 1 à 100.

    \item Créez un programme qui affiche les lettres de l'alphabet en utilisant une boucle \lstinline{for}.

    \item Écrivez un programme qui prend une liste de nombres et affiche chaque nombre multiplié par 2.

    \item Écrivez un programme qui affiche tous les éléments d'une liste de noms en commençant par la première lettre de chaque nom en majuscule.

    \item Créez un programme qui affiche les 10 premiers multiples de 3.

    \item Écrivez un programme qui affiche les nombres de 10 à 1 dans l'ordre décroissant.

    \item Écrivez un programme qui affiche la table de multiplication de 7 (de 1 à 10).

    \item Créez un programme qui demande à l'utilisateur un nombre entier, puis affiche les \lstinline{n} premiers nombres impairs, où \lstinline{n} est l'entier donné.

    \item Écrivez un programme qui prend une phrase et affiche chaque mot de cette phrase séparément.

    \item Créez un programme qui demande à l'utilisateur une chaîne de caractères et qui affiche chaque caractère de la chaîne individuellement.

    \item Écrivez un programme qui utilise une boucle \lstinline{for} pour compter le nombre de voyelles dans une phrase donnée.

    \item Créez un programme qui utilise une boucle \lstinline{for} pour calculer la factorielle d'un nombre donné par l'utilisateur.

    \item Écrivez un programme qui prend une liste de notes (entiers) et calcule la moyenne de ces notes.

    \item Créez un programme qui demande à l'utilisateur un nombre et affiche les diviseurs de ce nombre.

    \item Écrivez un programme qui affiche les nombres de 1 à 50, mais pour les multiples de 3 affiche "Fizz", pour les multiples de 5 affiche "Buzz", et pour les multiples de 3 et 5 affiche "FizzBuzz".

    \item Créez un programme qui affiche les nombres de la suite de Fibonacci jusqu'au 10\textsuperscript{e} terme.

\end{enumerate}
%%%%%%%%%%%%%%%%%%%%%%

