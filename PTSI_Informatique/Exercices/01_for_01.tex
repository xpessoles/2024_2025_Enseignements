
\begin{enumerate}

    \item Écrire une fonction \lstinline{question_01(n:int) -> None} qui affiche les nombres de 1 à $n$ en utilisant une boucle \lstinline{for}.

    \item Écrire une fonction \lstinline{question_02(n:int) -> None} qui affiche tous les nombres pairs de 1 à $n$.

    \item Écrire une fonction \lstinline{question_03(n:int) -> None} qui affiche les carrés des nombres de 1 à $n$.

    \item Écrire une fonction \lstinline{question(fruits:[str]) -> None} qui affiche les éléments d'une liste de fruits \lstinline{['pomme', 'banane', 'orange', 'raisin']}.

    \item Écrire une fonction \lstinline{question() -> } qui calcule et affiche la somme des nombres de 1 à 100.

    \item Écrire une fonction \lstinline{question() -> } qui affiche les lettres de l'alphabet en utilisant une boucle \lstinline{for}.

    \item Écrire une fonction \lstinline{question() -> } qui prend une liste de nombres et affiche chaque nombre multiplié par 2.

    \item Écrire une fonction \lstinline{question() -> } qui affiche tous les éléments d'une liste de noms en commençant par la première lettre de chaque nom en majuscule.

    \item Créer une fonction \lstinline{question() -> } qui affiche les 10 premiers multiples de 3.

    \item Écrire une fonction \lstinline{question() -> } qui affiche les nombres de 10 à 1 dans l'ordre décroissant.

    \item Écrire une fonction \lstinline{question() -> } qui affiche la table de multiplication de 7 (de 1 à 10).

    \item Écrire une fonction \lstinline{question() -> } qui demande à l'utilisateur un nombre entier, puis affiche les \lstinline{n} premiers nombres impairs, où \lstinline{n} est l'entier donné.

    \item Écrire une fonction \lstinline{question() -> } qui prend une phrase et affiche chaque mot de cette phrase séparément.

    \item Écrire une fonction \lstinline{question() -> } qui demande à l'utilisateur une chaîne de caractères et qui affiche chaque caractère de la chaîne individuellement.

    \item Écrire une fonction \lstinline{question() -> } qui utilise une boucle \lstinline{for} pour compter le nombre de voyelles dans une phrase donnée.

    \item Créer une fonction \lstinline{question() -> } qui utilise une boucle \lstinline{for} pour calculer la factorielle d'un nombre donné par l'utilisateur.

    \item Écrire une fonction \lstinline{question() -> } qui prend une liste de notes (entiers) et calcule la moyenne de ces notes.

    \item Créer une fonction \lstinline{question() -> } qui demande à l'utilisateur un nombre et affiche les diviseurs de ce nombre.

    \item Écrire une fonction \lstinline{question() -> } qui affiche les nombres de 1 à 50, mais pour les multiples de 3 affiche "Fizz", pour les multiples de 5 affiche "Buzz", et pour les multiples de 3 et 5 affiche "FizzBuzz".

    \item Écrire une fonction \lstinline{question() -> } qui affiche les nombres de la suite de Fibonacci jusqu'au 10\textsuperscript{e} terme.

\end{enumerate}
%%%%%%%%%%%%%%%%%%%%%%

