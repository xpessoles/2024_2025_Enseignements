
\begin{enumerate}

    \item Écrire une fonction \lstinline{question_01(n:int) -> None} qui affiche les nombres de 1 à $n$ en utilisant une boucle \lstinline{for}.

    \item Écrire une fonction \lstinline{question_02(n:int) -> None} qui affiche tous les nombres pairs de 1 à $n$.

    \item Écrire une fonction \lstinline{question_03(n:int) -> None} qui affiche les carrés des nombres de 1 à $n$.

    \item Écrire une fonction \lstinline{question_04(fruits:[str]) -> None} qui affiche un élément d'une liste de fruits \lstinline{['pomme', 'banane', 'orange', 'raisin']} par ligne.

    \item Écrire une fonction \lstinline{question_05(n:int) -> int} qui calcule et affiche la somme des nombres de 1 à n.

    \item Écrire une fonction \lstinline{question_06(mot:str) -> None} qui énumère les lettres du mot en utilisant une boucle \lstinline{for}.

%%%%%
    \item Écrire une fonction \lstinline{question_07(L:[int]) -> [int] } qui prend une liste de nombres et renvoie une liste ou chaque nombre est multiplié par 2.

    \item Écrire une fonction \lstinline{question_08(noms:[str]) -> [str]} qui prend une liste de noms et renvoie une autre liste de noms où chacun commence par une lettre capitale\sidenote{\lstinline{ch.capitalize()} permet de mettre la première lettre de la chaîne ch en capitale.}.

    \item Créer une fonction \lstinline{question_09(n:int) -> None} qui affiche les $n$ premiers multiples de 3.

    \item Écrire une fonction \lstinline{question_10(n:int) -> None} qui affiche les nombres de $n$ à 1 dans l'ordre décroissant.

    \item Écrire une fonction \lstinline{question_11(n:int) -> None} qui affiche la table de multiplication de $n$ (de 1 à 10).

    \item Écrire une fonction \lstinline{question_12(n:int) -> None} qui affiche les \lstinline{n} premiers nombres impairs, où \lstinline{n} est l'entier donné.

    \item Écrire une fonction \lstinline{question_13(phrase:str) -> None} qui prend une phrase en argument et affiche un mot par ligne.

    \item Écrire une fonction \lstinline{question_14(chaine:str) -> None} qui prend une chaîne de caractères en argument et qui affiche chaque caractère de la chaîne individuellement.

    \item Écrire une fonction \lstinline{question_15(phrase:str) -> int} qui utilise une boucle \lstinline{for} pour compter le nombre de voyelles dans une phrase donnée.

    \item Créer une fonction \lstinline{question_16(n:int) -> int} qui utilise une boucle \lstinline{for} pour calculer la factorielle d'un nombre donné par l'utilisateur.

    \item Écrire une fonction \lstinline{question_17(L:[int]) -> float} qui prend une liste de notes (entiers) et calcule la moyenne de ces notes en utilisant une boucle \lstinline{for}.

    \item Créer une fonction \lstinline{question_18(n:int) -> None} qui prend en argument un nombre et affiche les diviseurs de ce nombre.

    \item Écrire une fonction \lstinline{question_19(n:int) -> None} qui affiche les nombres de 1 à n, mais pour les multiples de 3 affiche "Fizz", pour les multiples de 5 affiche "Buzz", et pour les multiples de 3 et 5 affiche "FizzBuzz".

    \item Écrire une fonction \lstinline{question_20(n:int) -> None} qui affiche les nombres de la suite de Fibonacci jusqu'au n\textsuperscript{e} terme.

\end{enumerate}
%%%%%%%%%%%%%%%%%%%%%%

