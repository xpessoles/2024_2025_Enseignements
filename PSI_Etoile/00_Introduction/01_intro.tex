% ======= COURS ======= 
\renewcommand{\repExo}{\repRel/PSI_Cy_01_ModelisationSystemes/Ch_02_RevisionsSLCI/}
\renewcommand{\nomExo}{01_Modelisation_Perf_Fiche}
\graphicspath{{\repStyle/png}{\repExo\nomExo/images}}
%\input{\repExo\nomExo/\nomExo}
% ===================== 


\setchapterimage{Fond_SAF}
\setchapterpreamble[u]{\margintoc}

\chapter*{Bienvenue en PSI $\star$}
\setcounter{chapter}{1}

\begin{center}
\Large
\textsf{\textbf{Xavier PESSOLES}}
\normalsize 

\vspace{.5cm} 

\url{xpessoles@lamartin.fr} -- \url{https://xpessoles.github.io/}
\normalsize
\end{center}
\section{Travailler en SII}
\subsection{Les ressources}
\begin{itemize}
\item Site de classe : \url{psietoile.lamartin.fr} (ou \url{https://psietoilelamartin.github.io/}.
\item Site personnel : \url{https://xpessoles.github.io/}.
\item Annales corrigées de SI (jusqu'à 2021 ou 2022) : \url{https://www.upsti.fr/espace-etudiants/annales-de-concours}. 

\end{itemize}
\subsection{Connaître le cours}
Mes cours sont courts. Les méthodes, les résultats et les formules sont à connaître. 

\subsection{Maîtriser les applications directes, les calculs}
Vous êtes tous égaux devant la réalisation d'un calcul et devant la capacité à les mener vite et bien. 
Pour cela, il faut de \textbf{l’entraînement}. Cela vous permettra de gagner en efficacité, en assurance et donc de gagner des points au concours !

Pour les applications directes du concours, je vous propose des devoirs du soirs. Les corrigés (quand je les ai écrits) sont dispos sur le site de la classe. L'idée est d'en faire un par soir pour s'exercer à calculer le plus rapidement possible. Quelques points clés à maîtriser (liste non exhaustive) : 
\begin{itemize}
\item réaliser une fermeture de chaîne géométrique;
\item calculer \textbf{vite et sans faute} un produit vectoriel \textbf{en projetant que SI c'est indispensable};
\item dériver \textbf{vite et sans faute} un vecteur \textbf{en projetant que SI c'est indispensable};
\item calculer \textbf{vite et sans faute} une fonction de transfert en BO, en BF sous forme canonique;
\item exprimer \textbf{vite et sans faute} la sortie d'un système asservi avec une et avec deux entrées;
\item tracer \textbf{vite et sans faute} un diagramme de Bode;
\item \textit{etc.}
\end{itemize}

\subsection{S'entraîner en TD}
Je donne plus de TD qu'il est possible d'en faire en classe. J'essaye de faire des sujets différents dans les deux groupes. Si vous le souhaitez, il est aussi possible de faire des sous-groupes de besoin (dans un même groupe) pour différencier les besoins.
J'essaye de proposer des corrigés sur le site, mais ce n'est pas toujours le cas. N'hésitez pas à me solliciter s'il vous en faut davantage. 



\section{Le programme}
%Le programme de la filière MPSI/PCSI -- PSI est disponible \url{https://github.com/xpessoles/PSI/raw/master/Pedagogie/Programme-pcsi_sii.pdf}. 

Afin de vous évaluer ou de vous auto-évaluer je vous propose le découpage suivant.
\pagelayout{wide}

\begin{multicols}{2}
\allCompWideMP
\end{multicols}